\documentclass[11pt]{article}
\usepackage{trishnote}
\usepackage{tgpagella}  % for sample text

\titles{Overview Talk}
\author{\textbf{Notes By:} \textsc{Trishan Mondal} \\[0.15cm]
   \textbf{Speaker:} \textsc{Suresh Nayak} \\[0.15cm]
  \href{https://www.isibang.ac.in}{Indian Statistical Institute, Bangalore}}
\date{}

\begin{document}
\maketitle

\noindent We will begin with `cohomology of projective varieties' and we will see for smooth projective varieties many beautiful properties holds for de-Rahm cohomology, singular cohomology which don't get satiesfied for the case of `singular projective varieties'. We will discuss those with examples in this talk. 

\vspace*{0.2cm}

\noindent Let $X\subseteq \C P^N$ be a projective variety of dimension $n$ (in the sense of Krull dimension which will be same with the manifold dimension for the smooth case). $X$ is given by zeroes of some homogeneous polynomial thus it a closed subspace of $\C P^N$ and hence it is compact. For the smooth case $X$ is a `smooth manifold'(complex manifold). Some properties of smooth $X$ are described below, 

\begin{itemize}
    \item[$\circ$] X is given by zeroes of $g_1,\cdots,g_{N-n}$ with the rank of the matrix $\qty(\pdv{g_j}{z_i})_{ij}$ equal to $N-n$.
    \item[$\circ$] Hermitian metric on Tangent space of $X$.   
    \item[$\circ$] $X$ is an orientable manifold of dimension $2n$ admitting a Riemannian metric $g$ and a `complex structure' on it's Tangent space. 
    \item[$\circ$] There is also an alternating form $\omega$ (or K\"{a}hler differential). 
\end{itemize}

\section{Dualities}
We can compute the singular(simplicial) homology(cohomology) for $X$ with the coefficients in $\R$. Since $X$ is compact orientable manifold we can talk about the cup product pairing as follows: 
\[H^{i}(X;\R) \times H^{2n-i}(X;\R) \xrightarrow{\smile} H^{2n}(X;\R) \cong \R\]

\noindent is a `non-degenerate' pairing. Thus we have \textbf{Poincare Duality}, $$H^{2n-i}_{\text{Sing}}(X;\R) \cong H^{i}_{\text{Sing}}(X;\R)^* \cong H_i^{\text{sing}}(X;\R)$$

\noindent Since $X$ is a smooth manifold we can talk about de-Rahm cohomology. In a shofesticated language `de-Rahm cohomology is a cohomology of soft-resolution of constant sheaf'. In this case also we have the following as non-degenerate, 
\[H^{i}_{DR}(X;\R) \times H^{2n-i}_{DR}(X;\R) \xrightarrow{\wedge} H^{2n}_{DR}(X;\R) \xrightarrow[\int - \operatorname{vol-form}]{\sim} \R\]
Thus again we have the dualiy, $H^{2n-i}_{DR}(X;\R) \cong H^{i}_{DR}(X;\R)^*$. Connecting de-Rahm cohomology and singular(simplicial) cohomology with coefficients in $\R$, there is a beautiful theorem by de-Rahm stated as follows, 

\begin{Thm}{De-Rahm's Theorem}{}
     There is an isomorphism between the singular(simplicial) cohomology with coefficients in $\R$ and de-Rahm cohomology which is compatible with the product structure on both the V.S. 
\end{Thm}

\end{document}